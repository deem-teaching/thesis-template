% Some commands used in this file
\newcommand{\package}{\emph}

\chapter{Introduction}
\label{cha:introduction}

This chapter is a placeholder for the introduction of your thesis. 

The first paragraph of the introduction should describe the \emph{context}, followed by 1-3 paragraphs stating the \emph{problems} that are solved in this thesis.
The next paragraph should mention \emph{existing work} before introducing the \emph{idea} on how to solve the mentioned problems.\\


\textbf{Contributions:}
In the last paragraph list your contributions and outline the thesis as a list of bullet points containing a short introduction into the chapters. 

Additional information can be found here: \url{https://mboehm7.github.io/teaching/ws2122_isw/01_Introduction.pdf}, slide 21.


% Context (1 paragraph)

% Problems (1-3 paragraphs)

% [Idea (1 paragraph)]
This thesis proposes...

% Contributions (1 paragraph)
Detailed contributions include:

\begin{itemize}
    \item Integrated GPS tracking and satellite telemetry datasets to map penguin movements beyond traditional polar habitats.
    \item Conducted a comparative analysis across multiple species (e.g., King, Gentoo, and Little Blue Penguins) to explore differences in tropical dispersal behavior.
    \item Challenged prevailing assumptions that penguins are strictly cold-adapted species by proposing an ecological framework for thermal adaptability and behavioral plasticity.
\end{itemize}

